\section*{Aufgabe 9 - Das Gewichtspolynom eines linearen Codes}
\addcontentsline{toc}{subsection}{Aufgabe 9 - Das Gewichtspolynom eines linearen Codes}
\begin{enumerate}
	\item 
		\begin{enumerate}
		\item
		Der Repetitionscode $R_5$ besitzt nur zwei Codewörter, $(0 0 0
		0 0)$ und $(1 1 1 1 1)$, somit ist sein Gewichtspolynom
		\[ W_{R_5}(z) = z^5 + 1 \]
		Die Codewörter seines dualen Codes $R_5^{\bot}$ sind die
		Binärdarstellungen der Input-Wörter mit einem Paritätsbit vorne
		angehängt (1 bei ungeradem, 0 bei geradem
		\textsc{Hamming}-Gewicht des Input-Wortes). Daraus lässt sich
		ableiten, dass
		\[ W_{R_5^{\bot}}(z) = \sum_{\substack{0\leq j\leq 4\\ j
		\textnormal{ ungerade}}}\binom{4}{j}z^{j+1} +
		\sum_{\substack{0\leq j\leq 4\\ j \textnormal{
		  gerade}}}\binom{4}{j}z^j = 5z^4 + 10z^2 + 1 \]
		\item
		Aus der Codeworttabelle lässt sich das Polynom ablesen:

		\begin{center}
		\begin{tabular}{c|c|c}
		Input & Codewort & \textsc{Hamming}-Gewicht \\
		\hline
		$[0 0 0]$ & $[0 0 0 0 0]$ & $0$ \\
		$[0 0 1]$ & $[1 1 1 1 1]$ & $5$ \\
		$[0 1 0]$ & $[0 0 1 1 0]$ & $2$ \\
		$[0 1 1]$ & $[1 1 0 0 1]$ & $3$ \\
		$[1 0 0]$ & $[1 1 1 0 0]$ & $3$ \\
		$[1 0 1]$ & $[0 0 0 1 1]$ & $2$ \\
		$[1 1 0]$ & $[1 1 0 1 0]$ & $3$ \\
		$[1 1 1]$ & $[0 0 1 0 1]$ & $2$
		\end{tabular}
		\end{center}

		Das Gewichtspolynom ist somit:
		\[ W_{\mathcal{C}}(z) = z^5 + 3z^3 + 3z^2 + 1 \]
		Der Kern der Matrix $G$ ist
		\[ H=\begin{bmatrix}1&0&1&1&1\\1&1&0&0&0\end{bmatrix} \]
		Dies ist die Generatormatrix von $\mathcal{C}^{\bot}$. Dieser
		Code hat folgende Codeworttabelle:

		\begin{center}
		\begin{tabular}{c|c|c}
		Input & Codewort & \textsc{Hamming}-Gewicht \\
		\hline
		$[0 0]$ & $[0 0 0 0 0]$ & $0$ \\
		$[0 1]$ & $[1 1 0 0 0]$ & $2$ \\
		$[1 0]$ & $[1 0 1 1 1]$ & $4$ \\
		$[1 1]$ & $[0 1 1 1 1]$ & $4$
		\end{tabular}
		\end{center}

		und damit das Gewichtspolynom
		\[ W_{\mathcal{C}^{\bot}}(z) = 2z^4 + z^2 + 1 \]
		\item
		Das Gewichtspolynom des \textsc{Hamming}-Codes und seines
		dualen Codes lässt sich leicht aus der \textsc{Fano}-Ebene
		ableiten.
		$H(7,4)$ besteht aus den $7$ Geraden mit
		\textsc{Hamming}-Gewicht $3$ und deren $7$ Komplemente	mit
		\textsc{Hamming}-Gewicht $4$. Dazu kommen noch der Null- und
		Einsvektor. Folglich:
		\[ W_{H(7,4)}(z) = z^7 + 7z^4 + 7z^3 + 1 \]
		Der duale Code ist über die $7$ Komplemente der Geraden und dem
		Nullvektor definiert:
		\[ W_{S(7,3)}(z) = 7z^4 + 1 \]
		\end{enumerate}
	\item
		Die Wahrscheinlichkeit, dass ein Fehler bei der
		\textit{Fehlererkennung} auftritt ist
		\[ P_{err} = \sum_{c'\in\mathcal{C}} P(c'|c) =
		\sum_{e=c'-c\in\mathcal{C}} P(e|0) = \sum_{e\in\mathcal{C}}
		\prod_{1\leq j\leq n} p(e_j|0) \]
		Für einen $q$-ären symmetrischen Kanal ist das Produkt über die
		$p(e_j|0)$:
		\[ \prod_{1\leq j\leq n} p(e_j|0) = \left(\frac{p}{q-1}\right)^{\left
		\Vert e\right \|}\cdot(1-p)^{n-\left \Vert e \right \|} =
		(1-p)^n\cdot\left(\frac{p}{(q-1)(1-p)}\right)^{\left \Vert e
		\right \|} \]
		Das Gewichtspolynom lässt sich auch folgendermaßen schreiben:
		\[ W_{\mathcal{C}}(z) = \sum_{c\in\mathcal{C}} z^{\left \Vert c
		\right \|} = \left(\sum_{\substack{c\in\mathcal{C}\\c\not=0}}
		z^{\left \Vert c \right \|}\right) + 1 \]
		Daraus folgt unmittelbar
		\[ P_{err} =
		(1-p)^n\sum_{\substack{e\in\mathcal{C}\\e\not=0}}\left(\frac{p}{(q-1)(1-p)}\right)^{\left
		\Vert e \right \|} = (1-p)^n\cdot(W_\mathcal{C}(\delta)-1) \]
		mit $\delta = \frac{p}{(q-1)(1-p)}$.
\end{enumerate}
