\section*{Aufgabe 19 - Decodierung mittels Hadamard-Transformation}
\addcontentsline{toc}{subsection}{Aufgabe 19 - Decodierung mittels Hadamard-Transformation}
\begin{enumerate}
	\item
	Die ML-Decodierung aus Aufgabe 17 lässt sich auch so schreiben:
	\[ argmax_{c_{1},c_{2},\dots,c_{n}\in\mathcal{C}}\sum_{j=1}^{n}
	(-1)^{u\cdot g_{j}}\mu(y_{j}) \]
	Dies ist genau das Ergebnis der Multiplikation $\hk \cdot \mu(y)$, da
	die Einträge von $\mu$ wie die Spalten von $\hk$ indiziert sind und die
	Hadamard-Matrix an der Stelle $(u, v)$ genau den Eintrag
	$(-1)^{u\cdot v}$ besitzt. Alle Einträge von $\mu$, für deren Index $v$
	keine Spalte in der Generatormatrix existiert, sind $0$.
	\item
	\begin{itemize}
	\item Multiplikation $\hk \cdot \mu(y)$: $k \cdot 2^{k}$ Additionen
	\item $n$ Subtraktionen, um $\mu(y)$ zu erstellen (da $G$ genau $n$
	Spalten hat, bzw. natürlich für jedes $y_{j}$ das entsprechende
	$\mu(y_{j})$ berechnet werden muss)
	\item $2^{k} - 1$ Vergleiche, um das Maximum herauszusuchen
	\end{itemize}
	\item
	\lstset{ %
		language=Mathematica,
		basicstyle=\small,
		numbers=left,
		numberstyle=\small,
		numbersep=-9pt
	}
	\begin{lstlisting}
	mu[y_] := Log[2, P[[y, 1]]/P[[y, 2]]];

	Mu[y_, G_] :=
	  Flatten[
	   Table[
	    Sum[mu[y[[j]] + 1],
	     {j, Position[
	       Transpose[G],
	       IntegerDigits[i, 2, Length[G]]
	       ]}
	     ], {i, 0, 2^Length[G] - 1}]];

	Dml[y_, G_] :=
	  Module[{M = HadamardMultiply[Mu[y, G]]},
	   IntegerDigits[
	    Position[M, Max[M]] - 1, 2, Length[G]]
	   ];
	\end{lstlisting}
	Hierbei ist $P$ die entsprechende Kanalmatrix, $G$ die Generatormatrix.
\end{enumerate}
