\section*{Aufgabe 18 - Hadamard-Matrizen}
\addcontentsline{toc}{subsection}{Aufgabe 18 - Hadamard-Matrizen}
\begin{enumerate}
\item
	Da die Hadamard-Matrix über die Vektoren aus $\mathds{B}^k$ indiziert
	ist, gilt für $\hkp$, dass $2^k$ neue Indizes in beiden
	Dimensionen hinzukommen, die genau die Indizes von $\hk$ mit
	einer vorangestellten $1$ (die ursprünglichen Indizes werden um eine
	$0$ vorangestellten $1$ (die ursprünglichen Indizes werden um eine $0$
	erweitert).
	
	Die Skalarprodukte der Indizes sind also in den drei $2^{k} \times
	2^{k}$-Untermatrizen von $\hkp$ identisch mit denjenigen
	aus $\hk$, bei denen nicht sowohl im Zeilen- als auch im
	Spaltenindex eine $1$ voranstehen. In dieser letzten Untermatrix ist
	jedes Skalarprodukt genau um $1$ größer, was für die Einträge der
	Untermatrix eine Negation bedeutet.
\item
	\lstset{ %
		language=Mathematica,
		basicstyle=\small,
		numbers=left,
		numberstyle=\small,
		numbersep=-9pt
	}
	\begin{lstlisting}
   HadamardMultiply[x_] :=
      Module[{d, i},
         d = HadamardMultiplyRec[x];
         Return[
            Table[
            (* Index j binaer von hinten gelesen, 
	       'Butterfly'-Algorithmus *)
               i = 1 +(* Mathematica beginnt Indizierung bei 1 *)
                  FromDigits[
                   Reverse[
                    IntegerDigits[j - 1, 2, Log[2, Length[d]]]], 2];
               d[[i]],
               {j, 1, Length[d]}
            ]
         ]
      ];

   HadamardMultiplyRec[x_] :=
   (* prinzipiell FFT-Algorithmus ohne Verwendung von Einheitswurzeln *)
      Module[{c, g, u},
         If[Length[x] == 1,
            (* "then" *)
            x[[1]],
            (* "else" *)
            g = HadamardMultiplyRec[Take[x, {1, -1, 2}]];
	                           (* x1, x3, x5, ... *)
            u = HadamardMultiplyRec[Take[x, {2, -1, 2}]];
	                           (* x2, x4, x6, ... *)
            c = Flatten[{g + u, g - u}];
            (* 2^k Additionen/Subtraktionen pro Ebene im Rekursionsbaum, 
	       Log[2,2^k]=k Ebenen, insgesamt also k * 2^k 
	       Additionen/Subtraktionen *)
            Return[c];
         ]
      ];
	\end{lstlisting}
\item
	Aus der Konstruktion von $\hk$ ist offensichtlich dass
	$\hk = \hk^{t}$. Dann folgt:
	\[ \hkp^{2} = \begin{bmatrix}2\hk^{2} & \hk^{2} + \hk\cdot(-\hk) \\
	\hk^{2} + \hk\cdot(-\hk) & \hk^{2} + (-\hk)^{2}\end{bmatrix} =
	\begin{bmatrix}2\hk^{2} & [0] \\ [0] & 2\hk^{2}\end{bmatrix} \]
	Wegen
	\[ \mathcal{H}_{1}^{2} = \begin{bmatrix}2&0\\0&2\end{bmatrix} \]
	gilt wie behauptet
	\[ \hk^{2} = \begin{bmatrix}2^{k}\begin{bmatrix}1&0\\0&1\end{bmatrix} &
	\bigg[\ \  0\ \ \bigg] \\[1.0em] \bigg[\ \ 0\ \ \bigg] & 2^{k}\begin{bmatrix}1&0\\0&1\end{bmatrix}\end{bmatrix} = 2^{k}
	\mathds{1}_{2^{k}}\]
\end{enumerate}
