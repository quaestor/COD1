\section*{Aufgabe 14 - Gewichtspolynome der Simplexcodes und von Hamming-Codes}
\addcontentsline{toc}{subsection}{Aufgabe 14 - Gewichtspolynome der Simplexcodes und von Hamming-Codes}
\begin{enumerate}
\item
	Die \textsc{Reed-Muller}-Codes $RM(1,m)$ haben minimale
	\textsc{Hamming}-Distanz $2^{m-1}$ (siehe Blätter zu
	\textsc{Reed-Muller}-Codes). Da der Einsvektor bei diesen Codes ein
	gültiges Codewort ist (eine Zeile der Generatormatrix besteht nur aus
	Einsen) enthalten diese Codes nur den Nullvektor, den Einsvektor und
	Wörter mit \textsc{Hamming}-Gewicht genau gleich $2^{m-1}$.

	Aus den $RM(1,m)$ lassen sich nun auf einfache Weise die Simplexcodes
	ableiten, indem man aus den Wörtern von $RM(1,m)$ alle Wörter die mit
	einer $0$ beginnen wählt und diese $0$ streicht, was leicht aus der
	Generatormatrix ersichtlich ist:

	\[ G_{RM(1,m)} = \begin{bmatrix}1&\begin{matrix}1&\dots&1\end{matrix}\\
	\begin{matrix}0\\\vdots\\0\end{matrix}&H_{m} \end{bmatrix}\]

	(Die Kontrollmatrix $H_{m}$ des \textsc{Hamming}-Codes ist gleichzeitig
	die Generatormatrix des zugehörigen Simplexcodes.)

	Da bei diesem Vorgang der Einsvektor verloren geht, aber bei den
	anderen Codevektoren nur eine $0$ entfernt wird, bleibt das
	\textsc{Hamming}-Gewicht von $2^{m-1}$ erhalten.

	Somit haben die Simplexcodes nur noch ein Wort vom Gewicht $0$ und
	$2^{m} - 1$ Wörter vom Gewicht $2^{m-1}$, was für das Gewichtspolynom
	bedeutet:

	\[ W_{S_{m}} = 1 + (2^{m} - 1)z^{2^{m-1}} \]
\item
	Bestimmung des Gewichtspolynoms des $[15,11,3]$-\textsc{Hamming}-Codes
	mithilfe der \textsc{MacWilliams}-Identität in Mathematica:
	\lstset{ %
		language=Mathematica,
		basicstyle=\small,
		numbers=left,
		numberstyle=\small,
		numbersep=-9pt
	}
	\begin{lstlisting}
	n := 2^m - 1;
	WSm[x_] := 1 + (2^m - 1)*x^(2^(m - 1));
	WHm[z_] := 2^(-m)*(1 + z)^(2^m - 1)*WSm[(1 - z)/(1 + z)];

	m = 4;
	Together[WHm[z]] // Expand
	\end{lstlisting}
	\begin{align*}
	W_{H(15,11)} & = 1 + 35 z^3 + 105 z^4 + 168 z^5 + 280 z^6 + 435 z^7 + 435 z^8 + \\
	& 280 z^9 + 168 z^{10} + 105 z^{11} + 35 z^{12} + z^{15}
	\end{align*}

	Selbiges für den $[31,25,3]$-\textsc{Hamming}-Code:
	\begin{lstlisting}[firstnumber=7]
	m = 5;
	Together[WHm[z]] // Expand
	\end{lstlisting}
	\begin{align*}
	W_{H(31,25)} & = 1 + 155 z^3 + 1085 z^4 + 5208 z^5 + 22568 z^6 + 82615 z^7 +  \\
	& 247845 z^8 + 628680 z^9 + 1383096 z^{10} + 2648919 z^{11} + \\
	& 4414865 z^{12} + 6440560 z^{13} + 8280720 z^{14} + 9398115 z^{15} +  \\
	& 9398115 z^{16} + 8280720 z^{17} + 6440560 z^{18} + 4414865 z^{19} +  \\
	& 2648919 z^{20} + 1383096 z^{21} + 628680 z^{22} + 247845 z^{23} +  \\
	& 82615 z^{24} + 22568 z^{25} + 5208 z^{26} + 1085 z^{27} + 155 z^{28} + z^{31}
	\end{align*}
\end{enumerate}
