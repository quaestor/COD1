\section*{Aufgabe 22 - Überdeckungsradius}
\addcontentsline{toc}{subsection}{Aufgabe 22 - Überdeckungsradius}
\begin{enumerate}
\item
	Die \textsc{Hamming}-Codes sind $1$-perfekt. Ihr Überdeckungsradius ist
	somit $\rho = 1$.
\item
	folgt aus (3.), denn die $y = H x$ mit $x$ aus der Menge der
	\textit{coset leader} überdecken - da $H$ vollen Rang hat -
	$\mathds{F}_q^{n-k}$. Jeder Vektor in $\mathds{F}_q^{n-k}$ lässt sich
	also durch eine Linearkombination von $n - k$ Spalten aus $H$
	darstellen. $x$ kann somit höchstens Gewicht $n - k$ haben.
\item
	Aus der Nearest-Codeword-Decodierungsregel mittels \textit{coset
	leader} ($y \mapsto y - e, \quad e$ \textit{coset leader}) folgt, dass
	das Gewicht von $e$ der Abstand zum nächsten Codewort ist.

	Der \textit{coset leader} mit maximalem Gewicht kennzeichnet somit den
	größten Abstand eines Elementes aus $\mathds{F}^n$ von einem Codewort
	aus $\mathcal{C}$.
\item
	Dass jedes $y \in \mathds{F}_q^{n-k}$ sich durch eine Linearkombination
	von $r \leq \rho$ Spalten von $H$ darstellen lässt, wurde bereits in (2.)
	geschildert.
	
	Ist nun $y \in \mathds{F}_q^{n-k}$ Linearkombination von $r$ Spalten
	aus $H$, so gibt es ein $x \in \mathds{F}_q^n$ mit $y = H x$, sodass
	$d(x + c, c) = r\ \forall\ c \in \mathcal{C}$.
	Kann man $y$ als Linearkombination von höchstens $r$ Spalten aus $H$
	darstellen, so gilt $\rho \leq r$.
\item
	Dies folgt direkt aus 2. Die Minimaldistanz von MDS-Codes ist $d = n -
	k + 1$, für lineare Codes gilt $\rho \leq n - k$.
\item
	$\mathcal{C}$ besteht aus Vektoren $\left(u(\alpha_1), u(\alpha_2),
	\dots, u(\alpha_n)\right) \in \mathds{F}_q^n$, wobei $u(x)$ Polynom vom
	Grad $< k$.

	Sei $v(x)$ Polynom vom Grad $k$, $\underline{v} = \left(v(\alpha_1),
	\dots, v(\alpha_n)\right) \not\in \mathcal{C}$. Angenommen $d(\underline{u},
	\underline{v}) < n - k$, d.h. $u(\alpha_j) = v(\alpha_j)$ für mindestens $k +
	1$ Stellen $\alpha_j$: Interpolationspolynom eindeutig!
	\begin{align*} \Rightarrow\ & u(x) = v(x) \\
		\Rightarrow\ & d(\underline{u}, \underline{v}) \geq n - k
		\Rightarrow \rho = n - k
	\end{align*}
\end{enumerate}
