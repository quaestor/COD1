\section*{Aufgabe 24 - Verkürzte Codes und MDS-Codes}
\addcontentsline{toc}{subsection}{Aufgabe 24 - Verkürzte Codes und MDS-Codes}
\begin{enumerate}
	\item
		\begin{itemize}
		\item $n' = n - 1$, da offensichtlich jedes Codewort im neuen
		Code um genau eine Stelle kürzer ist.
		\item $k' \in \{k, k-1\}$, 
		\item $d' \geq d$, denn die Stelle, an der verkürzt wird, trägt
		nicht mehr zur Minimaldistanz bei und da die Anzahl der
		Codewörter höchstens verringert wird besteht die Möglichkeit,
		dass die Minimaldistanz größer wird.
		\end{itemize}
	\item
		Aus (1.) folgt, dass beim Verkürzen $d' \geq d$. Ein verkürzter
		MDS-Code ist damit von der Form $[n-1, k', d']$ mit $d' \geq d
		= n - k + 1$. Da $n - k \leq (n - 1) - k' + 1 \leq n - k + 1$ muss $k' = k
		- 1$ und $d' = d$ gelten.
\end{enumerate}
